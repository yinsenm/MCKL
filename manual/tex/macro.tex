% ============================================================================
%  MCKL/manual/tex/macro.tex
% ----------------------------------------------------------------------------
%  MCKL: Monte Carlo Kernel Library
% ----------------------------------------------------------------------------
%  Copyright (c) 2013-2017, Yan Zhou
%  All rights reserved.
%
%  Redistribution and use in source and binary forms, with or without
%  modification, are permitted provided that the following conditions are met:
%
%    Redistributions of source code must retain the above copyright notice,
%    this list of conditions and the following disclaimer.
%
%    Redistributions in binary form must reproduce the above copyright notice,
%    this list of conditions and the following disclaimer in the documentation
%    and/or other materials provided with the distribution.
%
%  THIS SOFTWARE IS PROVIDED BY THE COPYRIGHT HOLDERS AND CONTRIBUTORS "AS IS"
%  AND ANY EXPRESS OR IMPLIED WARRANTIES, INCLUDING, BUT NOT LIMITED TO, THE
%  IMPLIED WARRANTIES OF MERCHANTABILITY AND FITNESS FOR A PARTICULAR PURPOSE
%  ARE DISCLAIMED. IN NO EVENT SHALL THE COPYRIGHT HOLDER OR CONTRIBUTORS BE
%  LIABLE FOR ANY DIRECT, INDIRECT, INCIDENTAL, SPECIAL, EXEMPLARY, OR
%  CONSEQUENTIAL DAMAGES (INCLUDING, BUT NOT LIMITED TO, PROCUREMENT OF
%  SUBSTITUTE GOODS OR SERVICES; LOSS OF USE, DATA, OR PROFITS; OR BUSINESS
%  INTERRUPTION) HOWEVER CAUSED AND ON ANY THEORY OF LIABILITY, WHETHER IN
%  CONTRACT, STRICT LIABILITY, OR TORT (INCLUDING NEGLIGENCE OR OTHERWISE)
%  ARISING IN ANY WAY OUT OF THE USE OF THIS SOFTWARE, EVEN IF ADVISED OF THE
%  POSSIBILITY OF SUCH DAMAGE.
% ============================================================================

\overfullrule=1ex

\UseAbbr{aes}
\UseAbbr{ars}
\UseAbbr{avx}
\UseAbbr{batch}
\UseAbbr{blas}
\UseAbbr{brng}
\UseAbbr{cdf}
\UseAbbr{cmake}
\UseAbbr{cpb}
\UseAbbr{cpe}
\UseAbbr{cpu}
\UseAbbr{crtp}
\UseAbbr{ess}
\UseAbbr{fma}
\UseAbbr{gcc}
\UseAbbr{gnu}
\UseAbbr{icc}
\UseAbbr{ieee}
\UseAbbr{intel}
\UseAbbr{lapack}
\UseAbbr{llvm}
\UseAbbr{mckl}
\UseAbbr{mcmc}
\UseAbbr{mkl}
\UseAbbr{msvc}
\UseAbbr{nan}
\UseAbbr{nasm}
\UseAbbr{odr}
\UseAbbr{pdf}
\UseAbbr{pmf}
\UseAbbr{posix}
\UseAbbr{rdrand}
\UseAbbr{rng}
\UseAbbr{simd}
\UseAbbr{single}
\UseAbbr{smc}
\UseAbbr{smp}
\UseAbbr{std}
\UseAbbr{stl}
\UseAbbr{tbb}
\UseAbbr{tls}
\UseAbbr{ulp}
\UseAbbr{unix}
\UseAbbr{vmf}
\UseAbbr{vml}
\UseAbbr{vsl}

\UseAbbr[\aesni][\textsc]{aes-ni}
\UseAbbr[\cpp][\textcase]{C++}
\UseAbbr[\hdf]{hdf5}
\UseAbbr[\ilp]{ilp64}
\UseAbbr[\lp]{lp64}
\UseAbbr[\testu][\lnfigures]{TestU01}

\UseMathBB{I}

\UseMathCal{N}
\UseMathCal{X}

\MakeShortVerb{\|}

\def\rmin{r_{\mathrm{min}}}
\def\rmax{r_{\mathrm{max}}}

\def\xobs{X_{\mathrm{obs}}}
\def\xpos{X_{\mathrm{pos}}}
\def\xvel{X_{\mathrm{vel}}}
\def\yobs{Y_{\mathrm{obs}}}
\def\ypos{Y_{\mathrm{pos}}}
\def\yvel{Y_{\mathrm{vel}}}

\def\system{Two \cpu and operation system combinations are tested. The first is
  an Intel Xeon~W3550, running at~3.06~GHz base frequency and~3.33~GHz maximum
  turbo boost frequency, with four physical cores and eight logical ones
  through hyperthreading, with CentOS (verison~7.2) and the second is an Intel
  Core~i7-4960\textsc{hq}, running at~2.60~Ghz base frequency and~3.80~GHz
  maximum turbo boost freqency, with four physical cores and eight logical ones
  through hyperthreading, with macOS Sierra (version~10.12.1). They are labeled
  as ``Nehalem'' and ``Haswell'', respectively, based on their \cpu
  microarchitecture.}

\def\compiler{Three compilers are tested on each system, \llvm clang
  (version~3.9.1 on CentOS and version Apple~8.0.0 on macOS), \gnu \gcc
  (version~6.2.0 on bosth systems), and Intel \cpp compiler (version~17.0.0 on
  CentOS and version~17.0.1 on macOS).}

\def\compilerone{\compiler{} The results shown here is with the \llvm clang
  compiler.}

\def\compilerthree{\compiler{} They are labeled as ``\llvm'', ``\gnu'' and
  ``\intel'', respectively.}

\ExplSyntaxOn
\def\perftable#1#2#3#4{
  \file_if_exist:nT{tab/random_#1_#2_sse2.tex}{
    \begin{table}
      \input{tab/random_#1_#2_sse2}
      \caption{Performance~of~#3~(Nehalem)}
      \label{tab:Performance~of~#4~(Nehalem)}
    \end{table}
  }
  \file_if_exist:nT{tab/random_#1_#2_avx2.tex}{
    \begin{table}
      \input{tab/random_#1_#2_avx2}
      \caption{Performance~of~#3~(Haswell)}
      \label{tab:Performance~of~#4~(Haswell)}
    \end{table}
  }
}
\ExplSyntaxOff
