% ============================================================================
%  MCKL/manual/tex/macro.tex
% ----------------------------------------------------------------------------
%  MCKL: Monte Carlo Kernel Library
% ----------------------------------------------------------------------------
%  Copyright (c) 2013-2017, Yan Zhou
%  All rights reserved.
%
%  Redistribution and use in source and binary forms, with or without
%  modification, are permitted provided that the following conditions are met:
%
%    Redistributions of source code must retain the above copyright notice,
%    this list of conditions and the following disclaimer.
%
%    Redistributions in binary form must reproduce the above copyright notice,
%    this list of conditions and the following disclaimer in the documentation
%    and/or other materials provided with the distribution.
%
%  THIS SOFTWARE IS PROVIDED BY THE COPYRIGHT HOLDERS AND CONTRIBUTORS "AS IS"
%  AND ANY EXPRESS OR IMPLIED WARRANTIES, INCLUDING, BUT NOT LIMITED TO, THE
%  IMPLIED WARRANTIES OF MERCHANTABILITY AND FITNESS FOR A PARTICULAR PURPOSE
%  ARE DISCLAIMED. IN NO EVENT SHALL THE COPYRIGHT HOLDER OR CONTRIBUTORS BE
%  LIABLE FOR ANY DIRECT, INDIRECT, INCIDENTAL, SPECIAL, EXEMPLARY, OR
%  CONSEQUENTIAL DAMAGES (INCLUDING, BUT NOT LIMITED TO, PROCUREMENT OF
%  SUBSTITUTE GOODS OR SERVICES; LOSS OF USE, DATA, OR PROFITS; OR BUSINESS
%  INTERRUPTION) HOWEVER CAUSED AND ON ANY THEORY OF LIABILITY, WHETHER IN
%  CONTRACT, STRICT LIABILITY, OR TORT (INCLUDING NEGLIGENCE OR OTHERWISE)
%  ARISING IN ANY WAY OUT OF THE USE OF THIS SOFTWARE, EVEN IF ADVISED OF THE
%  POSSIBILITY OF SUCH DAMAGE.
% ============================================================================

\SetFontScale{mono}{footnotesize}

\pagestyle{plain}

\makeatletter
\global\let\tikz@ensure@dollar@catcode=\relax
\makeatother

\UseAbbr{aes}
\UseAbbr{ars}
\UseAbbr{avx}
\UseAbbr{batch}
\UseAbbr{blas}
\UseAbbr{brng}
\UseAbbr{cdf}
\UseAbbr{cmake}
\UseAbbr{cpb}
\UseAbbr{cpe}
\UseAbbr{cpu}
\UseAbbr{crtp}
\UseAbbr{ess}
\UseAbbr{fma}
\UseAbbr{gcc}
\UseAbbr{gnu}
\UseAbbr{icc}
\UseAbbr{ieee}
\UseAbbr{intel}
\UseAbbr{lapack}
\UseAbbr{llvm}
\UseAbbr{mckl}
\UseAbbr{mcmc}
\UseAbbr{mkl}
\UseAbbr{msvc}
\UseAbbr{nan}
\UseAbbr{nasm}
\UseAbbr{odr}
\UseAbbr{pdf}
\UseAbbr{pmf}
\UseAbbr{posix}
\UseAbbr{rdrand}
\UseAbbr{rng}
\UseAbbr{simd}
\UseAbbr{single}
\UseAbbr{smc}
\UseAbbr{smp}
\UseAbbr{std}
\UseAbbr{stl}
\UseAbbr{tbb}
\UseAbbr{tls}
\UseAbbr{ulp}
\UseAbbr{unix}
\UseAbbr{vmf}
\UseAbbr{vml}
\UseAbbr{vsl}

\UseAbbr[\aesni][\textsc]{aes-ni}
\UseAbbr[\cpp][\textcase]{C++}
\UseAbbr[\hdf]{hdf5}
\UseAbbr[\ilp]{ilp64}
\UseAbbr[\lp]{lp64}
\UseAbbr[\testu][\textlnum]{TestU01}

\UseMathBB{I}

\UseMathCal{N}
\UseMathCal{X}

\usepackage{shortvrb}
\MakeShortVerb{\|}

\NewDocumentCommand\rmin{}{r_{\mathrm{min}}}
\NewDocumentCommand\rmax{}{r_{\mathrm{max}}}

\NewDocumentCommand\xobs{}{X_{\mathrm{obs}}}
\NewDocumentCommand\xpos{}{X_{\mathrm{pos}}}
\NewDocumentCommand\xvel{}{X_{\mathrm{vel}}}
\NewDocumentCommand\yobs{}{Y_{\mathrm{obs}}}
\NewDocumentCommand\ypos{}{Y_{\mathrm{pos}}}
\NewDocumentCommand\yvel{}{Y_{\mathrm{vel}}}

\NewDocumentCommand\system{}{The performance is measured on an MacBook~Pro with
an Intel Core~i7--4960\Abbr{hq} \cpu running macOS Sierra (version 10.12.1).}

\NewDocumentCommand\compiler{}{Three compilers are tested, \llvm clang (version
  Apple 8.0.0), \gnu \gcc (version 6.3.0), and Intel \cpp compiler (version
17.0.1).}

\NewDocumentCommand\compilerone{}{\compiler{} Results of the \llvm clang
compiler is shown here.}

\NewDocumentCommand\compilerthree{}{\compiler{} They are labeled as ``\llvm'',
``\gnu'' and ``\intel'', respectively.}

\ExplSyntaxOn
\NewDocumentCommand\perftable{m m m m}{
  \file_if_exist:nT{tab/random_#1_#2_avx2.tex}{
    \begin{table}
      \input{tab/random_#1_#2_avx2}
      \caption{Performance~of~#3~(Haswell)}
      \label{tab:Performance~of~#4}
    \end{table}
  }
}
\ExplSyntaxOff
