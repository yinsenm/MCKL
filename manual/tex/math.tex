% ============================================================================
%  MCKL/manual/tex/math.tex
% ----------------------------------------------------------------------------
%  MCKL: Monte Carlo Kernel Library
% ----------------------------------------------------------------------------
%  Copyright (c) 2013-2016, Yan Zhou
%  All rights reserved.
%
%  Redistribution and use in source and binary forms, with or without
%  modification, are permitted provided that the following conditions are met:
%
%    Redistributions of source code must retain the above copyright notice,
%    this list of conditions and the following disclaimer.
%
%    Redistributions in binary form must reproduce the above copyright notice,
%    this list of conditions and the following disclaimer in the documentation
%    and/or other materials provided with the distribution.
%
%  THIS SOFTWARE IS PROVIDED BY THE COPYRIGHT HOLDERS AND CONTRIBUTORS "AS IS"
%  AND ANY EXPRESS OR IMPLIED WARRANTIES, INCLUDING, BUT NOT LIMITED TO, THE
%  IMPLIED WARRANTIES OF MERCHANTABILITY AND FITNESS FOR A PARTICULAR PURPOSE
%  ARE DISCLAIMED. IN NO EVENT SHALL THE COPYRIGHT HOLDER OR CONTRIBUTORS BE
%  LIABLE FOR ANY DIRECT, INDIRECT, INCIDENTAL, SPECIAL, EXEMPLARY, OR
%  CONSEQUENTIAL DAMAGES (INCLUDING, BUT NOT LIMITED TO, PROCUREMENT OF
%  SUBSTITUTE GOODS OR SERVICES; LOSS OF USE, DATA, OR PROFITS; OR BUSINESS
%  INTERRUPTION) HOWEVER CAUSED AND ON ANY THEORY OF LIABILITY, WHETHER IN
%  CONTRACT, STRICT LIABILITY, OR TORT (INCLUDING NEGLIGENCE OR OTHERWISE)
%  ARISING IN ANY WAY OUT OF THE USE OF THIS SOFTWARE, EVEN IF ADVISED OF THE
%  POSSIBILITY OF SUCH DAMAGE.
% ============================================================================

\chapter{Mathematical functions}
\label{chap:Mathemtical functions}

\section{Constants}
\label{sec:Constants}

\mckl defines some mathematical constants in the form of constant expression
functions. For example, to get the value of $\pi$ with a desired precision, one
can use the following,
\begin{verbatim}
constexpr float pi_f = const_pi<float>();
constexpr double pi_d = const_pi<double>();
constexpr long double pi_l = const_pi<long double>();
\end{verbatim}
The compiler evaluates these values at compile-time and thus there is no
performance difference from hard-coding the constants in the program, while the
readability is improved. All defined constants are listed in
table~\ref{tab:Mathematical constants}.

\begin{table}
  \begin{tabularx}{\textwidth}{p{1.25in}Lp{1.25in}L}
    \toprule
    Function & Value & Function & Value \\
    \midrule
    \texttt{const\_inf}            & $\infty$        &
    \texttt{const\_nan}            & NaN             \\
    \texttt{const\_zero}           & $0$             &
    \texttt{const\_one}            & $1$             \\
    \texttt{const\_pi}             & $\pi$           &
    \texttt{const\_pi\_2}          & $2\pi$          \\
    \texttt{const\_pi\_inv}        & $1/\pi$         &
    \texttt{const\_pi\_sqr}        & $\pi^2$         \\
    \texttt{const\_pi\_by2}        & $\pi/2$         &
    \texttt{const\_pi\_by3}        & $\pi/3$         \\
    \texttt{const\_pi\_by4}        & $\pi/4$         &
    \texttt{const\_pi\_by6}        & $\pi/6$         \\
    \texttt{const\_pi\_2by3}       & $2\pi/3$        &
    \texttt{const\_pi\_3by4}       & $3\pi/4$        \\
    \texttt{const\_pi\_4by3}       & $4\pi/3$        &
    \texttt{const\_sqrt\_pi}       & $\sqrt{\pi}$    \\
    \texttt{const\_sqrt\_pi\_2}    & $\sqrt{2\pi}$   &
    \texttt{const\_sqrt\_pi\_inv}  & $\sqrt{1/\pi}$  \\
    \texttt{const\_sqrt\_pi\_by2}  & $\sqrt{\pi/2}$  &
    \texttt{const\_sqrt\_pi\_by3}  & $\sqrt{\pi/3}$  \\
    \texttt{const\_sqrt\_pi\_by4}  & $\sqrt{\pi/4}$  &
    \texttt{const\_sqrt\_pi\_by6}  & $\sqrt{\pi/6}$  \\
    \texttt{const\_sqrt\_pi\_2by3} & $\sqrt{2\pi/3}$ &
    \texttt{const\_sqrt\_pi\_3by4} & $\sqrt{3\pi/4}$ \\
    \texttt{const\_sqrt\_pi\_4by3} & $\sqrt{4\pi/3}$ &
    \texttt{const\_ln\_pi}         & $\ln{\pi}$      \\
    \texttt{const\_ln\_pi\_2}      & $\ln{2\pi}$     &
    \texttt{const\_ln\_pi\_inv}    & $\ln{1/\pi}$    \\
    \texttt{const\_ln\_pi\_by2}    & $\ln{\pi/2}$    &
    \texttt{const\_ln\_pi\_by3}    & $\ln{\pi/3}$    \\
    \texttt{const\_ln\_pi\_by4}    & $\ln{\pi/4}$    &
    \texttt{const\_ln\_pi\_by6}    & $\ln{\pi/6}$    \\
    \texttt{const\_ln\_pi\_2by3}   & $\ln{2\pi/3}$   &
    \texttt{const\_ln\_pi\_3by4}   & $\ln{3\pi/4}$   \\
    \texttt{const\_ln\_pi\_4by3}   & $\ln{4\pi/3}$   &
    \texttt{const\_e}              & $\EE$           \\
    \texttt{const\_e\_inv}         & $1/\EE$         &
    \texttt{const\_sqrt\_e}        & $\sqrt{\EE}$    \\
    \texttt{const\_sqrt\_e\_inv}   & $\sqrt{1/\EE}$  &
    \texttt{const\_sqrt\_2}        & $\sqrt{2}$      \\
    \texttt{const\_sqrt\_3}        & $\sqrt{3}$      &
    \texttt{const\_sqrt\_5}        & $\sqrt{5}$      \\
    \texttt{const\_sqrt\_10}       & $\sqrt{10}$     &
    \texttt{const\_sqrt\_1by2}     & $\sqrt{1/2}$    \\
    \texttt{const\_sqrt\_1by3}     & $\sqrt{1/3}$    &
    \texttt{const\_sqrt\_1by5}     & $\sqrt{1/5}$    \\
    \texttt{const\_sqrt\_1by10}    & $\sqrt{1/10}$   &
    \texttt{const\_ln\_2}          & $\ln{2}$        \\
    \texttt{const\_ln\_3}          & $\ln{3}$        &
    \texttt{const\_ln\_5}          & $\ln{5}$        \\
    \texttt{const\_ln\_10}         & $\ln{10}$       &
    \texttt{const\_ln\_inv\_2}     & $1/\ln{2}$      \\
    \texttt{const\_ln\_inv\_3}     & $1/\ln{3}$      &
    \texttt{const\_ln\_inv\_5}     & $1/\ln{5}$      \\
    \texttt{const\_ln\_inv\_10}    & $1/\ln{10}$     &
    \texttt{const\_ln\_ln\_2}      & $\ln\ln{2}$     \\
    \bottomrule
  \end{tabularx}
  \caption{Mathematical constants}
  \label{tab:Mathematical constants}
\end{table}

\clearpage

\section{Vectorized functions}
\label{sec:Vectorized functions}

\mckl provides a set of vectorized elementary mathematical functions. For
example, to perform additions of two vectors,
\begin{verbatim}
size_t n = 1000;
Vector<double> a(n), b(n), y(n);
// Fill vectors a and b
add(n, a.data(), b.data(), y.data());
\end{verbatim}
This is equivalent to,
\begin{verbatim}
for (size_t i = 0; i != n; ++i)
    y[i] = a[i] + b[i];
\end{verbatim}
The functions defined are listed in tables~\ref{tab:Arithmetic functions}
to~\ref{tab:Rounding functions}. For each function, the first parameter is
always the length of the vector, and the last is a pointer to the output vector
(except for |sincos| and |modf|, which have two output parameters). For all
functions, the output parameters are always vectors. If there are more than one
input parameters, then some of them, but not all, can be scalars. For example,
for the function call |muladd(n, a, b, c, y)| in table~\ref{tab:Arithmetic
functions}, the input parameters are |a|, |b|, and |c|. Some of them, not all,
can be scalars instead of pointers to vectors. The output parameter |y| has to
be a pointer to a vector. Therefore, there are seven versions of this function
for each type of the parameters.

The input of these functions  can be either real numbers (|double|, etc.), or
complex numbers (|std::complex<double>|, etc.), or both. The supported data
types are also listed in the tables. In most cases, output data type is the
same as the input. There are a few exceptions. The |abs| and |arg| functions
always have real numbers as output. The |cis| function takes real numbers as
input and complex numbers as output. Note that, mixed precision is not allowed.
For example,
\begin{verbatim}
Vector<double> a(n);
Vector<double> b(n);
Vector<double> y(n);
muladd(n, a.data(), b.data(), 2, y.data());
\end{verbatim}
causes compile-time error because the fourth argument is an integer while the
others are floating points. The correct call shall be,
\begin{verbatim}
muladd(n, a.data(), b.data(), 2.0, y.data());
\end{verbatim}
The same principle applies to mixed types functions (|abs|, |arg| and |cis|).
That is, a |double| input can only be paired with a |std::complex<double>|
output for |cis| and \emph{vice versa} for |abs| and |arg|.

With only the standard library, these functions do not provide performance
advantage compared to simple loops. When \mkl \vml is available, some functions
can have substantial performance improvement when all input arguments are
vectors of types |float| or |double|, or their complex counterparts. The
performance of vectorized random number generating introduced later in
section~\ref{sec:Vectorized random number generating} heavily depends on these
functions.

\begin{table}
  \begin{tabularx}{\textwidth}{LLL}
    \toprule
    Function & Operation & Data type \\
    \midrule
    \texttt{add(n, a, b, y)}       & $y = a + b$           & Real, Complex \\
    \texttt{sub(n, a, b, y)}       & $y = a - b$           & Real, Complex \\
    \texttt{sqr(n, a, y)}          & $y = a^2$             & Real          \\
    \texttt{mul(n, a, b, y)}       & $y = ab$              & Real, Complex \\
    \texttt{mulbyconj(n, a, b, y)} & $y = a\bar{b}$        & Complex       \\
    \texttt{conj(n, a, y)}         & $y = \bar{a}$         & Complex       \\
    \texttt{abs(n, a, y)}          & $y = \Abs{a}$         & Real, Complex \\
    \texttt{arg(n, a, y)}          & $y = \mathrm{arg}(a)$ & Complex       \\
    \texttt{muladd(n, a, b, c, y)} & $y = ab + c$          & Real, Complex \\
    \texttt{fma(n, a, b, c, y)}    & $y = ab + c$          & Real          \\
    \bottomrule
  \end{tabularx}
  \caption{Arithmetic functions}
  \label{tab:Arithmetic functions}
\end{table}

\begin{table}
  \begin{tabularx}{\textwidth}{LLL}
    \toprule
    Function & Operation & Data type \\
    \midrule
    \texttt{inv(n, a, y)}      & $y = 1 / a$            & Real          \\
    \texttt{div(n, a, b, y)}   & $y = a / b$            & Real, Complex \\
    \texttt{sqrt(n, a, y)}     & $y = \sqrt{a}$         & Real, Complex \\
    \texttt{invsqrt(n, a, y)}  & $y = 1 / \sqrt{a}$     & Real          \\
    \texttt{cbrt(n, a, y)}     & $y = \sqrt[3]{a}$      & Real          \\
    \texttt{invcbrt(n, a, y)}  & $y = 1 / \sqrt[3]{a}$  & Real          \\
    \texttt{pow2o3(n, a, y)}   & $y = a^{2/3}$          & Real          \\
    \texttt{pow3o2(n, a, y)}   & $y = a^{3/2}$          & Real          \\
    \texttt{pow(n, a, b, y)}   & $y = a^b$              & Real, Complex \\
    \texttt{hypot(n, a, b, y)} & $y = \sqrt{a^2 + b^2}$ & Real          \\
    \bottomrule
  \end{tabularx}
  \caption{Power and root functions}
  \label{tab:Power and root functions}
\end{table}

\begin{table}
  \begin{tabularx}{\textwidth}{LLL}
    \toprule
    Function & Operation & Data type \\
    \midrule
    \texttt{exp(n, a, y)}   & $y = \EE^a$       & Real, Complex \\
    \texttt{exp2(n, a, y)}  & $y = 2^a$         & Real          \\
    \texttt{expm1(n, a, y)} & $y = \EE^a - 1$   & Real          \\
    \texttt{log(n, a, y)}   & $y = \ln a$       & Real, Complex \\
    \texttt{log2(n, a, y)}  & $y = \log_2 a$    & Real          \\
    \texttt{log10(n, a, y)} & $y = \log_{10} a$ & Real, Complex \\
    \texttt{log1p(n, a, y)} & $y = \ln(a + 1)$  & Real          \\
    \bottomrule
  \end{tabularx}
  \caption{Exponential and logarithm functions}
  \label{tab:Exponential and logarithm functions}
\end{table}

\begin{table}
  \begin{tabularx}{\textwidth}{LLL}
    \toprule
    Function & Operation & Data type \\
    \midrule
    \texttt{cos(n, a, y)}       & $y = \cos(a)$        & Real, Complex \\
    \texttt{sin(n, a, y)}       & $y = \sin(a)$        & Real, Complex \\
    \texttt{sincos(n, a, y, z)} & $y = \sin(a)$, $z = \cos(a)$ & Real  \\
    \texttt{cis(n, a, y)}       & $y = \cos(a) + i\sin(a)$   & Complex \\
    \texttt{tan(n, a, y)}       & $y = \tan(a)$        & Real, Complex \\
    \texttt{acos(n, a, y)}      & $y = \arccos(a)$     & Real, Complex \\
    \texttt{asin(n, a, y)}      & $y = \arcsin(a)$     & Real, Complex \\
    \texttt{atan(n, a, y)}      & $y = \arctan(a)$     & Real, Complex \\
    \texttt{atan2(n, a, y)}     & $y = \arctan(a / b)$ & Real          \\
    \bottomrule
  \end{tabularx}
  \caption{Trigonometric functions}
  \label{tab:Trigonometric functions}
\end{table}

\begin{table}
  \begin{tabularx}{\textwidth}{LLL}
    \toprule
    Function & Operation & Data type \\
    \midrule
    \texttt{cosh(n, a, y)}  & $y = \cosh(a)$             & Real, Complex \\
    \texttt{sinh(n, a, y)}  & $y = \sinh(a)$             & Real, Complex \\
    \texttt{tanh(n, a, y)}  & $y = \tanh(a)$             & Real, Complex \\
    \texttt{acosh(n, a, y)} & $y = \mathrm{arc}\cosh(a)$ & Real, Complex \\
    \texttt{asinh(n, a, y)} & $y = \mathrm{arc}\sinh(a)$ & Real, Complex \\
    \texttt{atanh(n, a, y)} & $y = \mathrm{arc}\tanh(a)$ & Real, Complex \\
    \bottomrule
  \end{tabularx}
  \caption{Hyperbolic functions}
  \label{tab:Hyperbolic functions}
\end{table}

\begin{table}
  \begin{tabularx}{\textwidth}{LLL}
    \toprule
    Function & Operation & Data type \\
    \midrule
    \texttt{erf(n, a, y)}     & $y = \mathrm{erf}(a)$  & Real \\
    \texttt{erfc(n, a, y)}    & $y = \mathrm{erfc}(a)$ & Real \\
    \texttt{cdfnorm(n, a, y)} & $y = (1 + \mathrm{erf}(a / \sqrt{2})) / 2$
    & Real \\
    \texttt{lgamma(n, a, y)}  & $y = \ln\Gamma(a)$     & Real \\
    \texttt{tgamma(n, a, y)}  & $y = \Gamma(a)$        & Real \\
    \bottomrule
  \end{tabularx}
  \caption{Special functions}
  \label{tab:Special functions}
\end{table}

\begin{table}
  \begin{tabularx}{\textwidth}{LLL}
    \toprule
    Function & Operation & Data type \\
    \midrule
    \texttt{floor(n, a, y)} & $y = \Floor{a}$                       & Real \\
    \texttt{ceil(n, a, y)}  & $y = \Ceil{a}$                        & Real \\
    \texttt{trunc(n, a, y)} & $y = \mathrm{sgn}(a)\Floor{\Abs{a}}$  & Real \\
    \texttt{round(n, a, y)} & $y = \text{nearest integer of }a$     & Real \\
    \texttt{nearbyint(n, a, y)} & $y = \text{nearest integer of }a$ & Real \\
    \texttt{rint(n, a, y)}  & $y = \text{nearest integer of }a$     & Real \\
    \texttt{modf(n, a, y, z)} &
    $y = \mathrm{sign}(a)\Floor{\Abs{a}}$, $z = a - y$ & Real \\
    \bottomrule
  \end{tabularx}
  \caption{Rounding functions}
  \label{tab:Rounding functions}
\end{table}

\clearpage

\section{Fused multiplication and addition}
\label{sec:Fused multiplication and addition}

The |muladd| and |fma| functions in the last section differs in that |fma|
always does fused multiplication and addition through |std::fma| while |muladd|
only does so when hardware support for \fma3 is detected. Also |muladd|
supports complex numbers as input while |fma| does not. However, the compiler
may or may not be able to make use the hardware support. And it may not
vectorize the loop as most modern \cpp compilers would do for simpler
operations such as addition and multiplication. When software implementation of
|std::fma| is used, it will be much slower than using a multiplication and an
addition. In this case, there are assembly implementations that take advantage
of the hardware support for |float| and |double| precision. To enable this
feature, one need to build and link to the runtime library
(see~ref{sec:Optional runtime library}). One also need to set the configuration
macro |MCKL_USE_ASM_LIB| to true. In addition, this feature is only enabled for
platforms with \fma3 instruction set support. The library detects the
availability of these instructions through compiler pre-defined macros. If this
mechanism is not adequate, one can manually enable or disable them through the
configuration macro |MCKL_HAS_FMA|. Note that, when the compiler is able to
generate vectorized loop for |std::fma|, the assembly library may or may not
outperform the compiler generated binary. If the compiler generated binary is
preferred but the runtime library is also enabled for its other features, then
one can define the configuration macro |MCKL_USE_ASM_FMA| to false to disable
this feature.

\section{Experimental vectorized implementation}
\label{sec:Experimental vectorized implementation}

Optionally, one can enable an experimental implementation of a few selected
functions, listed in table~\ref{tab:Functions with experimental vectorized
  implementation}, to enhance their performance of |double| precision. To
enable this feature, one need to build and link to the runtime library
(see~\ref{sec:Optional runtime library}). One also need to set the
configuration macros |MCKL_USE_ASM_LIB| and |MCKL_USE_ASM_VMF| to true. In
addition, this feature is only enabled for platforms with \avx2 and \fma3
instruction sets support. The library detects the availability of these
instructions through compiler pre-defined macros. If this mechanism is not
adequate, one can manually enable or disable them through the configuration
macros, |MCKL_HAS_AVX2| and |MCKL_HAS_FMA|, respectively.

\subsection{Caveats}
\label{sub:Caveats}

The four trigonometric functions have reduced ranges. They only give meaningful
results for input $\Abs{a} < A$ where $A \approx 2^{30.5}$. For finite values
with larger magnitudes, finite results will be given but the accuracy
degenerate quickly. For $A < \Abs{a} < 2^{43}$, there are about three accurate
decimal digits. For values with even larger magnitudes, the results are
useless.

The three exponential functions have slightly reduced upper bounds of ranges.

\subsection{Subnormal numbers}
\label{sub:Subnormal numbers}

Positive subnormal inputs for |log|, |log2| and |log10| will give $-\infty$ as
results. Subnormal inputs for |exp| and |exp2| will give $0$ as results. For
other functions, subnormal inputs are treated the same way as normal numbers.

\subsection{Special values}
\label{sub:Special values}

The handling of special values, $-0.0$, $0.0$, $-\infty$, $\infty$, \nan,
underflows, and overflows follows the \ieee~754 standard, except for the
treatment of subnormal numbers as described above.

\subsection{Measured accuracy}
\label{sub:Measured accuracy}

These functions have reduced accuracy compared to the standard library.
Reputable implementations of math functions usually give results with error
bounded by $1\ulp$. Some of these functions, especially the trigonometric
functions, give considerably larger errors. Tables~\ref{tab:Measured accuracy
  of vectorized implementation of exp} to~\ref{tab:Measured accuracy of
  vectorized implementation of tan} give the measured upper bounds of errors
and the intervals on which they are measured. In the tables, $\delta$ denotes
the smallest normal |double| precision floating numbers.

\begin{table}
  \begin{tabularx}{\textwidth}{LL}
    \toprule
    Function & Operation \\
    \midrule
    \texttt{exp(n, a, y)}       & $y = \EE^a$                   \\
    \texttt{exp2(n, a, y)}      & $y = 2^a$                     \\
    \texttt{expm1(n, a, y)}     & $y = \EE^a - 1$               \\
    \texttt{log(n, a, y)}       & $y = \ln a$                   \\
    \texttt{log2(n, a, y)}      & $y = \log_2 a$                \\
    \texttt{log10(n, a, y)}     & $y = \log_{10} a$             \\
    \texttt{log1p(n, a, y)}     & $y = \ln(a + 1)$              \\
    \texttt{cos(n, a, y)}       & $y = \cos(a)$                 \\
    \texttt{sin(n, a, y)}       & $y = \sin(a)$                 \\
    \texttt{sincos(n, a, y, z)} & $y = \sin(a)$, $z = \cos(a)$  \\
    \texttt{tan(n, a, y)}       & $y = \tan(a)$                 \\
    \bottomrule
  \end{tabularx}
  \caption{Functions with experimental vectorized implementation}
  \label{tab:Functions with experimental vectorized implementation}
\end{table}

\begin{table}
  \begin{tabularx}{\textwidth}{LL}
    \toprule
    Range & Error (\ulp) \\
    \midrule
    $[-708.4, -707)$      & $1$ \\
    $[-707, -500)$        & $1$ \\
    $[-500, -1)$          & $1$ \\
    $[-1, -\delta)$       & $1$ \\
    $[-\delta, \delta)$   & $0$ \\
    $[\delta, 1)$         & $1$ \\
    $[1, 500)$            & $1$ \\
    $[500, 707)$          & $1$ \\
    $[707, 709.44)$       & $1$ \\
    \bottomrule
  \end{tabularx}
  \caption{Measured accuracy of vectorized implementation of \texttt{exp}}
  \label{tab:Measured accuracy of vectorized implementation of exp}
\end{table}

\begin{table}
  \begin{tabularx}{\textwidth}{LL}
    \toprule
    Range & Error (\ulp) \\
    \midrule
    $[-1022, -1000)$      & $1$ \\
    $[-1000, -500)$       & $1$ \\
    $[-500, -1)$          & $1$ \\
    $[-1, -\delta)$       & $1$ \\
    $[-\delta, \delta)$   & $0$ \\
    $[\delta, 1)$         & $1$ \\
    $[1, 500)$            & $1$ \\
    $[500, 1000)$         & $1$ \\
    $[1000, 1022)$        & $1$ \\
    \bottomrule
  \end{tabularx}
  \caption{Measured accuracy of vectorized implementation of \texttt{exp2}}
  \label{tab:Measured accuracy of vectorized implementation of exp2}
\end{table}

\begin{table}
  \begin{tabularx}{\textwidth}{LL}
    \toprule
    Range & Error (\ulp) \\
    \midrule
    $[-708.4, -707)$         & $0$ \\
    $[-707, -500)$           & $0$ \\
    $[-500, -1)$             & $1$ \\
    $[-1, -\ln(2))$          & $1$ \\
    $[-\ln(2), -\ln(2) / 2)$ & $2$ \\
    $[-\ln(2) / 2, -\delta)$ & $1$ \\
    $[-\delta, \delta)$      & $0$ \\
    $[\delta, \ln(2) / 2)$   & $1$ \\
    $[\ln(2) / 2, \ln(2))$   & $2$ \\
    $[\ln(2), 1)$            & $1$ \\
    $[1, 500)$               & $1$ \\
    $[500, 707)$             & $1$ \\
    $[707, 709.44)$          & $1$ \\
    \bottomrule
  \end{tabularx}
  \caption{Measured accuracy of vectorized implementation of \texttt{expm1}}
  \label{tab:Measured accuracy of vectorized implementation of expm1}
\end{table}

\begin{table}
  \begin{tabularx}{\textwidth}{LL}
    \toprule
    Range & Error (\ulp) \\
    \midrule
    $[\delta, \sqrt{2} / 2)$  & $1$ \\
    $[\sqrt{2} / 2, sqrt{2})$ & $1$ \\
    $[\sqrt{2}, 10)$          & $1$ \\
    $[10, 10^2)$              & $1$ \\
    $[10^2, 10^3)$            & $1$ \\
    $[10^3, 10^4)$            & $1$ \\
    \bottomrule
  \end{tabularx}
  \caption{Measured accuracy of vectorized implementation of \texttt{log}}
  \label{tab:Measured accuracy of vectorized implementation of log}
\end{table}

\begin{table}
  \begin{tabularx}{\textwidth}{LL}
    \toprule
    Range & Error (\ulp) \\
    \midrule
    $[\delta, \sqrt{2} / 2)$  & $1$ \\
    $[\sqrt{2} / 2, sqrt{2})$ & $2$ \\
    $[\sqrt{2}, 10)$          & $1$ \\
    $[10, 10^2)$              & $1$ \\
    $[10^2, 10^3)$            & $1$ \\
    $[10^3, 10^4)$            & $1$ \\
    \bottomrule
  \end{tabularx}
  \caption{Measured accuracy of vectorized implementation of \texttt{log2}}
  \label{tab:Measured accuracy of vectorized implementation of log2}
\end{table}

\begin{table}
  \begin{tabularx}{\textwidth}{LL}
    \toprule
    Range & Error (\ulp) \\
    \midrule
    $[\delta, \sqrt{2} / 2)$  & $1$ \\
    $[\sqrt{2} / 2, sqrt{2})$ & $2$ \\
    $[\sqrt{2}, 10)$          & $1$ \\
    $[10, 10^2)$              & $1$ \\
    $[10^2, 10^3)$            & $1$ \\
    $[10^3, 10^4)$            & $1$ \\
    \bottomrule
  \end{tabularx}
  \caption{Measured accuracy of vectorized implementation of \texttt{log10}}
  \label{tab:Measured accuracy of vectorized implementation of log10}
\end{table}

\begin{table}
  \begin{tabularx}{\textwidth}{LL}
    \toprule
    Range & Error (\ulp) \\
    \midrule
    $[\delta, \sqrt{2} / 2)$  & $1$ \\
    $[\sqrt{2} / 2, sqrt{2})$ & $2$ \\
    $[\sqrt{2}, 10)$          & $1$ \\
    $[10, 10^2)$              & $1$ \\
    $[10^2, 10^3)$            & $1$ \\
    $[10^3, 10^4)$            & $1$ \\
    \bottomrule
  \end{tabularx}
  \caption{Measured accuracy of vectorized implementation of \texttt{log1p}}
  \label{tab:Measured accuracy of vectorized implementation of log1p}
\end{table}

\begin{table}
  \begin{tabularx}{\textwidth}{LL}
    \toprule
    Range & Error (\ulp) \\
    \midrule
    $[-1.7\times10^9, -1)$ & $2$ \\
    $[-1, -\delta)$        & $2$ \\
    $[-\delta, \delta)$    & $0$ \\
    $[\delta, 1)$          & $2$ \\
    $[1, 1.7\times10^9)$   & $2$ \\
    \bottomrule
  \end{tabularx}
  \caption{Measured accuracy of vectorized implementation of \texttt{cos}}
  \label{tab:Measured accuracy of vectorized implementation of cos}
\end{table}

\begin{table}
  \begin{tabularx}{\textwidth}{LL}
    \toprule
    Range & Error (\ulp) \\
    \midrule
    $[-1.7\times10^9, -1)$ & $2$ \\
    $[-1, -\delta)$        & $2$ \\
    $[-\delta, \delta)$    & $0$ \\
    $[\delta, 1)$          & $2$ \\
    $[1, 1.7\times10^9)$   & $2$ \\
    \bottomrule
  \end{tabularx}
  \caption{Measured accuracy of vectorized implementation of \texttt{sin}}
  \label{tab:Measured accuracy of vectorized implementation of sin}
\end{table}

\begin{table}
  \begin{tabularx}{\textwidth}{LL}
    \toprule
    Range & Error (\ulp) \\
    \midrule
    $[-1.7\times10^9, -1)$ & $2$ \\
    $[-1, -\delta)$        & $2$ \\
    $[-\delta, \delta)$    & $0$ \\
    $[\delta, 1)$          & $2$ \\
    $[1, 1.7\times10^9)$   & $2$ \\
    \bottomrule
  \end{tabularx}
  \caption{Measured accuracy of vectorized implementation of \texttt{sincos}}
  \label{tab:Measured accuracy of vectorized implementation of sincos}
\end{table}

\begin{table}
  \begin{tabularx}{\textwidth}{LL}
    \toprule
    Range & Error (\ulp) \\
    \midrule
    $[-1.7\times10^9, -1)$ & $2$ \\
    $[-1, -\delta)$        & $2$ \\
    $[-\delta, \delta)$    & $0$ \\
    $[\delta, 1)$          & $2$ \\
    $[1, 1.7\times10^9)$   & $2$ \\
    \bottomrule
  \end{tabularx}
  \caption{Measured accuracy of vectorized implementation of \texttt{tan}}
  \label{tab:Measured accuracy of vectorized implementation of tan}
\end{table}
