% ============================================================================
%  MCKL/manual/tex/random_distribution.tex
% ----------------------------------------------------------------------------
%  MCKL: Monte Carlo Kernel Library
% ----------------------------------------------------------------------------
%  Copyright (c) 2013-2017, Yan Zhou
%  All rights reserved.
%
%  Redistribution and use in source and binary forms, with or without
%  modification, are permitted provided that the following conditions are met:
%
%    Redistributions of source code must retain the above copyright notice,
%    this list of conditions and the following disclaimer.
%
%    Redistributions in binary form must reproduce the above copyright notice,
%    this list of conditions and the following disclaimer in the documentation
%    and/or other materials provided with the distribution.
%
%  THIS SOFTWARE IS PROVIDED BY THE COPYRIGHT HOLDERS AND CONTRIBUTORS "AS IS"
%  AND ANY EXPRESS OR IMPLIED WARRANTIES, INCLUDING, BUT NOT LIMITED TO, THE
%  IMPLIED WARRANTIES OF MERCHANTABILITY AND FITNESS FOR A PARTICULAR PURPOSE
%  ARE DISCLAIMED. IN NO EVENT SHALL THE COPYRIGHT HOLDER OR CONTRIBUTORS BE
%  LIABLE FOR ANY DIRECT, INDIRECT, INCIDENTAL, SPECIAL, EXEMPLARY, OR
%  CONSEQUENTIAL DAMAGES (INCLUDING, BUT NOT LIMITED TO, PROCUREMENT OF
%  SUBSTITUTE GOODS OR SERVICES; LOSS OF USE, DATA, OR PROFITS; OR BUSINESS
%  INTERRUPTION) HOWEVER CAUSED AND ON ANY THEORY OF LIABILITY, WHETHER IN
%  CONTRACT, STRICT LIABILITY, OR TORT (INCLUDING NEGLIGENCE OR OTHERWISE)
%  ARISING IN ANY WAY OUT OF THE USE OF THIS SOFTWARE, EVEN IF ADVISED OF THE
%  POSSIBILITY OF SUCH DAMAGE.
% ============================================================================

\chapter{Performance of Distributions}
\label{chap:Performance of Distributions}

\system

\compilerone

Four usage cases of the distributions are considered. First, if the
distribution is available in the standard library, we measure the following
case,
\begin{verbatim}
std::normal_distribution<double> rnorm_std(0, 1);
for (size_t i = 0; i = n; ++i)
    r[i] = rand(rng, rnorm_std);
\end{verbatim}
Second, we measure the performance of the library's implementation,
\begin{verbatim}
NormalDistribution<double> rnorm_mckl(0, 1);
for (size_t i = 0; i = n; ++i)
    r[i] = rand(rng, rnorm_mckl);
\end{verbatim}
The third and the fourth are the vectorized performance,
\begin{verbatim}
rand(rng, rnorm_mckl, n, r.data());
\end{verbatim}
See section~\ref{sec:Vectorized Random Number Generating}. The difference is
that we first test the performance without the \mkl \vml library and then with
it. When \mkl \vml is not used, the assembly implementation of selected
functions are used. For all the four above, the |ARS| \rng is used
(section~\ref{sub:AES Round Function Based Random Number Generators}). The last
is when there are \mkl routines for generating random numbers from the
distribution, either directly or indirectly. In this case, we use the
|MKL_ARS5| \rng (see section~\ref{sec:MKL Random Number Generators}),
\begin{verbatim}
MKL_ARS5 rng_mkl;
rand(rng_mkl, rnorm_mckl, n, r.data());
\end{verbatim}
In all cases, we repeat the simulations 100 times, each time with $n$ chosen
randomly between 5,000 and 10,000. The total number of cycles of the 100
simulations are recorded, and then divided by the total number of elements
generated. This gives the performance measurement in \cpe. This experiment is
repeated ten times, and the best results are shown. The five cases are labeled
``\std'', ``\mckl'', ``\vmf'', ``\vml'' and ``\mkl'', respectively, in tables
below.

\perftable{distribution}{u01}{standard uniform distributions}{%
standard uniform distributions}

\perftable{distribution}{inverse}{distributions using the inverse method}{%
distributions using the inverse method}

\perftable{distribution}{beta}{Beta distribution}{%
Beta distribution}

\perftable{distribution}{chisquared}{$\chi^2$ distribution}{%
chi-squared distribution}

\perftable{distribution}{gamma}{Gamma distribution}{%
Gamma distribution}

\perftable{distribution}{fisherf}{Fisher's $F$-distribution}{%
Fisher's F-distribution}

\perftable{distribution}{normal}{Normal and related distributions}{%
Normal and related distributions}

\perftable{distribution}{stable}{Stable distribution}{%
Stable distribution}

\perftable{distribution}{studentt}{Student's $t$-distribution}{%
Student's t-distribution}

\perftable{distribution}{int}{discrete distributions}{%
discrete distributions}
