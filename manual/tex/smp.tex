% ============================================================================
%  MCKL/manual/tex/smp.tex
% ----------------------------------------------------------------------------
%  MCKL: Monte Carlo Kernel Library
% ----------------------------------------------------------------------------
%  Copyright (c) 2013-2016, Yan Zhou
%  All rights reserved.
%
%  Redistribution and use in source and binary forms, with or without
%  modification, are permitted provided that the following conditions are met:
%
%    Redistributions of source code must retain the above copyright notice,
%    this list of conditions and the following disclaimer.
%
%    Redistributions in binary form must reproduce the above copyright notice,
%    this list of conditions and the following disclaimer in the documentation
%    and/or other materials provided with the distribution.
%
%  THIS SOFTWARE IS PROVIDED BY THE COPYRIGHT HOLDERS AND CONTRIBUTORS "AS IS"
%  AND ANY EXPRESS OR IMPLIED WARRANTIES, INCLUDING, BUT NOT LIMITED TO, THE
%  IMPLIED WARRANTIES OF MERCHANTABILITY AND FITNESS FOR A PARTICULAR PURPOSE
%  ARE DISCLAIMED. IN NO EVENT SHALL THE COPYRIGHT HOLDER OR CONTRIBUTORS BE
%  LIABLE FOR ANY DIRECT, INDIRECT, INCIDENTAL, SPECIAL, EXEMPLARY, OR
%  CONSEQUENTIAL DAMAGES (INCLUDING, BUT NOT LIMITED TO, PROCUREMENT OF
%  SUBSTITUTE GOODS OR SERVICES; LOSS OF USE, DATA, OR PROFITS; OR BUSINESS
%  INTERRUPTION) HOWEVER CAUSED AND ON ANY THEORY OF LIABILITY, WHETHER IN
%  CONTRACT, STRICT LIABILITY, OR TORT (INCLUDING NEGLIGENCE OR OTHERWISE)
%  ARISING IN ANY WAY OUT OF THE USE OF THIS SOFTWARE, EVEN IF ADVISED OF THE
%  POSSIBILITY OF SUCH DAMAGE.
% ============================================================================

\chapter{Symmetric multiprocessing}
\label{chap:Symmetric multiprocessing}

Many Monte Carlo algorithms admit straight forward parallelism. In this
chapter, we introduce how to use the library to parallelize a Monte Carlo
algorithm implementations on shared memory symmetric multiprocessing (\smp)
platforms.

\section{Decomposition of evaluation objects}
\label{sec:Decomposition of evaluation objects}

Recall that, in section~\ref{sec:Sampler} we introduced the concepts of sampler
evaluation objects. The primary method of iterating a sampler is through
one or more evaluation objects that is convertible to the following type,
\begin{verbatim}
using eval_type = std::function<void(size_t, Particle<T> &)>;
\end{verbatim}
In some situations, part of the evaluation process can be parallelized. For
example, consider the weight step in the simple particle filter in
section~\ref{sec:Example (PF)}. Recall the simple |PFWeight| function in
section~\ref{sub:Implementation (PF)},
\begin{verbatim}
void PFWeight(size_t iter, Particle<PF> &particle)
{
    Vector<double> weight(particle.size());
    for (auto idx : particle)
        weight[idx.i()] = idx.log_likelihood(iter);
    particle.weight().add_log(weight.data());
}
\end{verbatim}
The loop inside the function can be parallelized, since the assignment is
independent of each other. However, the statements before and after the loop
cannot be parallelized. Therefore, we can decompose it into three simple
functions. Below, we re-implement |PFWeight| as a class that is convertible to
|eval_type|.
\begin{verbatim}
class PFWeight
{
    public:
    void operator()(size_t iter, Particle<PF> &particle)
    {
        eval_first(iter, particle);
        for (auto idx : particle)
            eval_each(idx);
        eval_last(iter, particle);
    }

    void eval_first(size_t, Particle<PF> &particle)
    {
        weight_.resize(particle.size());
    }

    void eval_each(size_t iter, ParticleIndex<PF> idx)
    {
        weight_[idx.i()] = idx.log_likelihood(iter);
    }

    void eval_last(size_t, Particle<PF> &particle)
    {
        particle.weight().add_log(weight_.data());
    }

    private:
    Vector<double> weight_;
};
\end{verbatim}
Of course, such an implementation provides no benefits compared to the
original. In fact it is rather cumbersome. However, the library provides a
simple mechanism to convert any operators implemented this way to a
parallelized one. For example,
\begin{verbatim}
class PFWeightSMP : public SamplerEvalSMP<PF>
{
    public:
    void eval_first(size_t, Particle<PF> &particle)
    {
        weight_.resize(particle.size());
    }

    void eval_each(size_t iter, ParticleIndex<PF> idx)
    {
        weight_[idx.i()] = idx.log_likelihood(iter);
    }

    void eval_last(size_t, Particle<PF> &particle)
    {
        particle.weight().add_log(weight_.data());
    }

    private:
    Vector<double> weight_;
};
\end{verbatim}
This new implementation differs from the one earlier in two places. First, it
is now a derived class of |SamplerEvalSMP<PF>|. Second, the interface operator,
which makes the class type convertible to |eval_type| of |Sampler| is removed.
Now, the loop inside that operator in earlier implementations is parallelized.
Still, for this simple function, the performance benefits through
parallelization is minimal. However, in less trivial situations, |eval_each|
may be substantially more complicated. And the extra effort to decompose the
implementation into these three functions is minimal and there can be
considerable performance advantage.

\subsection{Mutations and weighting}
\label{sub:Mutations and weighting}

We have already seen the |SamplerEvalSMP| class template. The full declaration
of the template is as below,
\begin{verbatim}
template <typename T, typename = Virtual, typename = BackendSMP>
class SamplerEvalSMP;
\end{verbatim}
The first template parameter is the same as that of |Sampler|. The second
specifies the polymorphism method. The default is to use virtual functions. We
introduce another method later in section~\ref{sec:Static polymorphism}. The
last one selects the parallel programming models, or ``backends''. The default
is platform dependent and is detailed in section~\ref{sec:Programming models}.
In this section, we discuss the default behavior.

The class template has one public method.
\begin{verbatim}
void operator()(size_t iter, Particle<T> &particle);
\end{verbatim}
This makes instances of its derived classes convertible to |eval_type| of
|Sampler|. Its constructors and other special members are protected. And thus
it cannot be used on its own. It also has three other protected members,
\begin{verbatim}
virtual void eval_first(size_t, Particle<T> &);
virtual void eval_each(size_t, ParticleIndex<T>);
virtual void eval_last(size_t, Particle<T> &);
\end{verbatim}
The default implementation of each is to do nothing. They are not pure virtual
functions, and thus their implementation in the derived class is optional. The
public operator is implemented as if it is the same as we saw earlier in the
|PFWeight| class. How the loop is parallelized is implementation detail. The
only requirement is that the derived class implementation of |eval_each| has to
be thread-safe.

A full treatment of thread-safety is beyond the scope of this manual. Here are
rule of thumbs. In the standard library, this library and others designed with
thread-safety in mind, one can usually assume the following,
\begin{itemize}
  \item Namespace scope functions are thread-safe.
  \item Constant member functions are thread-safe.
  \item Non-constant member functions are \emph{potentially not} thread-safe.
\end{itemize}
Note that, constant member function is often a sufficient, but not necessary
condition for thread-safety. For example, the |eval_each| method itself is a
non-constant member function, but it is required to be thread-safe.

Here is one example of a problematic implementation of |eval_each|,
\begin{verbatim}
using T = StateMatrix<RowMajor, 1, double>;

class Eval : SamplerEvalSMP<T>
{
    public:
    void eval_each(size_t iter, ParticleIndex<T> idx) { idx(0) = normal_(rng_); }

    private:
    std::mt19937 rng_;
    std::normal_distribution<double> normal_;
};
\end{verbatim}
The methods |operator()| of \rng engines and the distribution generators are
generally non-constant members. And therefore they are potentially not
thread-safe. In fact, in the above example, they are indeed so. And when
parallelized the behavior is undefined. The correct way is to construct the
distribution object within |eval_each| and use the \rng engines in the particle
system,
\begin{verbatim}
class Eval : SamplerEvalSMP<T>
{
    public:
    void eval_each(size_t iter, ParticleIndex<T> idx)
    {
        std::normal_distribution<double> normal_;
        idx(0) = normal_(idx.rng());
    }
};
\end{verbatim}

\subsection{Estimation}
\label{sub:Estimation}

A function or class that is convertible to |eval_type| of |Monitor| can also be
parallelized. This is done through the following class template,
\begin{verbatim}
template <typename T, typename = Virtual, typename = BackendSMP>
class MonitorEvalSMP;
\end{verbatim}
This is similar to |SamplerEvalSMP| except for the |eval_each| method, which
has the following signature,
\begin{verbatim}
virtual void eval_each(size_t iter, size_t dim, ParticleIndex<T> idx, double *r);
\end{verbatim}
Now the output parameter |r| no longer points an $N$ by $d$ matrix. Instead it
points to a $d$-vector specific for each particle. For example, recall the
estimation function |PFEstimate| in
section~\ref{sub:Implementation (PF)}.
\begin{verbatim}
void PFEstimate(size_t, size_t, Particle<PF> &particle, double *r)
{
    for (auto idx : particle) {
        *r++ = idx.pos_x();
        *r++ = idx.pos_y();
    }
}
\end{verbatim}
This can be parallelized as the following,
\begin{verbatim}
class PFEstimate : public MonitorEvalSMP<PF>
{
    public:
    void eval_each(size_t, size_t, ParticleIndex<PF> idx, double *r)
    {
        *r++ = idx.pos_x();
        *r++ = idx.pos_y();
    }
};
\end{verbatim}
Optionally, the derived class can also have |eval_first| and |eval_last|
methods, similar to that of |SamplerEvalSMP|.

\section{Static polymorphism}
\label{sec:Static polymorphism}

Calling virtual functions has a small cost at runtime. There are situations
where there is a large number of particles, and thus parallelization is
beneficial, but the computation of |eval_each| is light weighted and the cost
of virtual function calls becomes significant. In this case, one can specify
the second template parameter of |SamplerEvalSMP| and |MonitorEvalSMP| to force
the compiler to statically dispatch calls of |eval_each|. For example,
\begin{verbatim}
class Eval : SamplerEvalSMP<T, Derived>;
\end{verbatim}
The class |Derived| is where the base class \emph{start} looking for methods
|eval_each|, etc. It does not have to implement these methods itself. For
example,
\begin{verbatim}
template <typename Derived>
class Eval : public SamplerEvalSMP<T, Derived>
{
    public:
    void eval_first(size_t iter, Particle<T> &particle);
};

class Eval1 : public Eval<Eval1>
{
    public:
    void eval_each(size_t iter, ParticleIndex<T> idx);
};

class Eval2 : public Eval<Eval2>
{
    public:
    void eval_each(size_t iter, ParticleIndex<T> idx);
};
\end{verbatim}
Here, |Eval| is a direct derived class of |SamplerEvalSMP|, and it implements
the |eval_first| method. This method is inherited by |Eval1| and |Eval2|, which
implement the |eval_each| method. The base class |SamlerEvalSMP| starts looking
for evaluation methods in |Eval1| and |Eval2|. The |eval_first| method is found
in |Eval|. The |eval_each| method is found in |Eval1| and |Eval2|. The
|eval_last| method is found in the base class.

The implementations of methods |eval_each| etc., can be either constant,
non-constant, or static. This type of use is called \emph{Curiously Recurring
  Template Pattern} (\crtp\footnote{
\url{https://en.wikipedia.org/wiki/Curiously_recurring_template_pattern}}). The
key point is that, the template parameter |Derived| tells |SamlerEvalSMP| where
should it start looking for |eval_each|, etc. And it casts itself to |Derived|
if it finds non-static member functions. It calls the methods directly if it
finds static member functions. The class template |SamplerEvalSMP| has
\emph{non-virtual} default implementations of these methods.

\section{Programming models}
\label{sec:Programming models}

The third template parameter of |SamplerEvalSMP| and |MonitorEvalSMP| specifies
the programming model to use for parallelization. Possible types are
\begin{verbatim}
class BackendSEQ; // Sequential implementation
class BackendSTD; // Using the standard library
class BackendOMP; // Using OpenMP
class BackendTBB; // Using Intel TBB
\end{verbatim}
It is also possible to implement new backends through partial specialization of
|SamplerEvalSMP| and |MonitorEvalSMP|. However, this is a too technical topic
to be included in this manual.

Each partial specialization of |SamplerEvalSMP| for the types above has a
protected method,
\begin{verbatim}
template <typename... Args>
void run(size_t iter, Particle<T> &particle, size_t grainsize, Args &&... args);
\end{verbatim}
which is called by the public operator method. One may use this function to
gain finer control of the parallelization. The behavior and allowable arguments
depend on the specific programming model. At the time of writing, this only has
effects on the \tbb backend. For others, these arguments are ignored.

In the case of \tbb, one can control the task division through the grain size.
For example,
\begin{verbatim}
class Eval : SamplerEvalSMP<T, Eval, BackendTBB>
{
    public:
    void operator()(size_t iter, Particle<T> &particle) { run(iter, particle, G); }

    void eval_each(size_t, ParticleIndex<T> idx);
};
\end{verbatim}
The argument $G$ passed to |run| above controls the threading such that, each
thread processes at least $G / 2$ particles. See the documents of \tbb for more
information of grain size. The additional arguments |args| are passed directly
as the optional arguments of |tbb::parallel_for|, the partitioner and the task
group context. See documents of \tbb for more details.

\section{Vectorization}
\label{sec:Vectorization}

It is often more efficient to process a vector as a whole instead of through a
loop. For example, considering the task of initializing states from a Gamma
distribution,
\begin{verbatim}
using T = StateMatrix<ColMajor, 1, double>;

class Eval
{
    public:
    void operator()(size_t iter, Particle<T> &particle)
    {
        std::gamma_distribution<double> gamma(2, 1);
        auto &rng = particle.rng();
        for (auto idx : particle)
            idx(0) = gamma(rng);
    }
};
\end{verbatim}
Later in section~\ref{sec:Vectorized random number generating} we show that the
library provides vectorized random numbers generating. For example, the above
can be written as,
\begin{verbatim}
class Eval
{
    public:
    void operator()(size_t iter, Particle<T> &particle)
    {
        GammaDistribution<double> gamma(2, 1);
        auto &rng = particle.rng();
        auto n = particle.size();
        auto r = particle.state().col_data(0);
        rand(rng, gamma, n, r);
    }
};
\end{verbatim}
which is about four to five times faster than the loop earlier. This is called
\emph{vectorization}. However, if the sample size is large, we would also like
to parallelize the loop. It is not possible to combine vectorization and
parallelization through the |eval_each| method we have seen. In this situation,
instead of implementing |eval_each|, one can implement an |eval_range| method.
For example,
\begin{verbatim}
class Eval : public SamplerEvalSMP<T>
{
    public:
    void eval_range(size_t iter, const ParticleRang<T> &range)
    {
        GammaDistribution<double> gamma(2, 1);
        auto &rng = range.begin().rng();
        auto n = range.size();
        auto r = range.particle().state().col_data(0) + range.first();
        rand(rng, gamma, n, r);
    }
};
\end{verbatim}
If the such a method is defined, the particle system is first partitioned into
multiple disjoint ranges. And each thread processes one of more of such ranges.
The |eval_range| method has to be thread-safe. If both |eval_range| and
|eval_each| are defined, the later is ignored.

The class |ParticleRange| is an abstraction of a subset of a particle system.
It can be constructed by the index of the first particle within the range and
one pass the index of the last particle within the range, together with a
pointer the particle system that it belongs to. For example,
\begin{verbatim}
ParticleRange<T> range(first, last, &particle);
\end{verbatim}
or alternatively,
\begin{verbatim}
auto range = particle.range(first, last);
\end{verbatim}
To access the particle system, use
\begin{verbatim}
range.particle();     // A reference to particle
range.particle_ptr(); // A pointer to particle
\end{verbatim}
It has the following methods,
\begin{verbatim}
range.first();
range.last();
range.begin(); // => range.particle().index(range.first());
range.end();   // => range.particle().index(range.last());
\end{verbatim}
It can also be iterated, similar to |Particle|. For example,
\begin{verbatim}
for (auto idx : range)
    /* process ParticleIndex<T> object idx */;
\end{verbatim}

Special care needs to be taken when using the \tbb backend and |eval_range|
together. The purpose of using it instead of |eval_each| is to process vectors
as a whole or to only execute some costly operations only once. For optimal
performance, the vectors cannot be too short. Otherwise, the performance is
worse than using a loop. It is important to control the grain size. Therefore,
it is recommended that if one does want to use |eval_range|, then one should
also redefine the interface operator. For example,
\begin{verbatim}
class Eval : public SamplerEvalSMP<T>
{
    public:
    Eval(std::size_t G) : G_(G) {}

    void operator()(size_t iter, Particle<T> &particle) { run(iter, particle, G_); }

    void eval_range(size_t iter, const ParticleRang<T> &range);

    private:
    std::size_t G_
};
\end{verbatim}
where $G$ is the grain size. Even though the grain size has no effect for
backends other than \tbb at the moment, it is planned to provide similar
functionality as \tbb for other backends in the future.

\section{Random number generating}
\label{sec:Random number generating}

As we have already mentioned in section~\ref{sub:Mutations and weighting}, care
has to be taken when generating random numbers in a multi-threaded environment.
The |Particle| class maintain a collection of \rng engines for this purpose.
For example,
\begin{verbatim}
void eval_each(size_t iter, ParticleIndex<T> &idx)
{
    auto &rng = idx.rng(); // => idx.particle().rng(idx.i());
    std::normal_distribution<double> normal(0, 1);
    idx(0) = normal(rng);
    idx(1) = normal(rng);
}
\end{verbatim}
Whenever possible, the library uses thread-local storage (\tls) to reduce
memory usage while still provides thread-safe of the above |rng|. It is more
efficient to obtain a reference to the thread-local object only once instead of
calling the |idx.rng()| method everything an \rng engine is needed. This is the
main reason that we did not use the alternative,
\begin{verbatim}
void eval_each(size_t iter, ParticleIndex<T> &idx)
{
    std::normal_distribution<double> normal(0, 1);
    idx(0) = normal(idx.rng());
    idx(1) = normal(idx.rng());
}
\end{verbatim}
Similarly, for |eval_range|, one can use the following,
\begin{verbatim}
void eval_range(size_t iter, const ParticleRang<T> &range)
{
    auto &rng = range.begin().rng();
    for (auto idx : range)
        /* use rng for idx */;
}
\end{verbatim}
Note that, all particles within the range are processed by a single thread, and
thus it is safe to use one \rng engine for all of them, unless the for loop
above is further parallelized, which may cause performance penalty due to over
submitting, and thus not recommended.
