% ============================================================================
%  MCKL/manual/tex/perf_random_distribution.tex
% ----------------------------------------------------------------------------
%  MCKL: Monte Carlo Kernel Library
% ----------------------------------------------------------------------------
%  Copyright (c) 2013-2016, Yan Zhou
%  All rights reserved.
%
%  Redistribution and use in source and binary forms, with or without
%  modification, are permitted provided that the following conditions are met:
%
%    Redistributions of source code must retain the above copyright notice,
%    this list of conditions and the following disclaimer.
%
%    Redistributions in binary form must reproduce the above copyright notice,
%    this list of conditions and the following disclaimer in the documentation
%    and/or other materials provided with the distribution.
%
%  THIS SOFTWARE IS PROVIDED BY THE COPYRIGHT HOLDERS AND CONTRIBUTORS "AS IS"
%  AND ANY EXPRESS OR IMPLIED WARRANTIES, INCLUDING, BUT NOT LIMITED TO, THE
%  IMPLIED WARRANTIES OF MERCHANTABILITY AND FITNESS FOR A PARTICULAR PURPOSE
%  ARE DISCLAIMED. IN NO EVENT SHALL THE COPYRIGHT HOLDER OR CONTRIBUTORS BE
%  LIABLE FOR ANY DIRECT, INDIRECT, INCIDENTAL, SPECIAL, EXEMPLARY, OR
%  CONSEQUENTIAL DAMAGES (INCLUDING, BUT NOT LIMITED TO, PROCUREMENT OF
%  SUBSTITUTE GOODS OR SERVICES; LOSS OF USE, DATA, OR PROFITS; OR BUSINESS
%  INTERRUPTION) HOWEVER CAUSED AND ON ANY THEORY OF LIABILITY, WHETHER IN
%  CONTRACT, STRICT LIABILITY, OR TORT (INCLUDING NEGLIGENCE OR OTHERWISE)
%  ARISING IN ANY WAY OUT OF THE USE OF THIS SOFTWARE, EVEN IF ADVISED OF THE
%  POSSIBILITY OF SUCH DAMAGE.
% ============================================================================

\chapter{TestU01 results}
\label{chap:TestU01 results}

All \rng{}s implemented in \mckl and those in the standard library are tested
with the \testu library~\cite{Lecuyer:2007hv}. The tests are performed as the
following, for each \rng, we initialize eight instances, each with a random
seed. The random stream feed into \testu batteries are such that, the first
thousand random integers come from the first instance of the \rng, then another
thousand from the second instance, and so on. After all eight instances are
used, we start with the first instance again. And repeat the process. This is
similar to the usage pattern of \rng{}s in a parallel program.

The method of generating standard uniform random numbers from random integers
affect the quality of the random stream feed into the \testu batteries as well
as the \rng itself. For each \rng we test five standard uniform distributions,
\begin{verbatim}
std::uniform_real_distribution<double>;
U01Distribution<double>;
U01CCDistribution<double>;
U01CODistribution<double>;
U01OCDistribution<double>;
U01OODistribution<double>;
\end{verbatim}
They are labeled as ``\std'', ``\textsc{u01}'', ``\textsc{u01cc}'',
``\textsc{u01co}'', ``\textsc{u01oc}'' and ``\textsc{u01oo}'', respectively, in
the tables below. See section~\ref{sec:Standard uniform distribution} for
details of the last five distributions. The configuration macro
|MCKL_U01_USE_FIXED_POINT| is set to false. The \std method uses
|std::generate_canonical|.

The SmallCrush, Crush and BigCrush batteries are applied to each \rng. If the
$p$-value of a test is within the range $(10^{-4}, 1 - 10^{-4})$, then it is
said to pass the test. If it is outside the range $(10^{-10}, 1 - 10^{-10})$,
then it is said to fail the test. Otherwise, we mark the results as suspicious.
In table~\ref{tab:TestU01 SmallCrush results} to~\ref{tab:TestU01 BigCrush
results}, we list the number of tests failed by each \rng. The number within
parentheses are the number of tests with suspicious $p$-values. For \rng{}s
without any observed failures or suspicious $p$-values, they are omitted from
the tables.

Note that, if an \rng passes considerably more tests with the \std method than
one of the \mckl distributions, then it is most likely to be some weakness of
the \rng at bit level randomness. As described in section~\ref{sec:Uniform bits
distribution}, the random integers are first combined into 32 or 64 random bits
before process by the standard uniform distributions. That is, for an \rng with
range $R = 2^V$, we assume that all $V$ bits are random.

One shall the choose the combination of \rng{}s and distributions carefully
based on the results in the tables below. In particular, almost all standard
library distributions make use of |std::generate_canonical|, while all
distributions in \mckl use one of the fixed point distribution. A rule of thumb
is to always use an \mckl \rng when using an \mckl distribution for better
performance when generating a large number of random numbers. They are much
faster than any standard library \rng with the same crush-resistant quality.

\newpage

\begin{table}
  \input{tab/random_testu01_smallcrush}%
  \caption{\testu SmallCrush results}
  \label{tab:TestU01 SmallCrush results}
\end{table}

\begin{table}
  \input{tab/random_testu01_crush}%
  \caption{\testu Crush results}
  \label{tab:TestU01 Crush results}
\end{table}

\begin{table}
  % \input{tab/random_testu01_bigcrush}%
  \caption{\testu BigCrush results}
  \label{tab:TestU01 BigCrush results}
\end{table}
